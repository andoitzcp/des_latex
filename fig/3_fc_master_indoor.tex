\documentclass{standalone}
\usepackage[utf8]{inputenc}
\usepackage[spanish]{babel}
\usepackage{tikz}
\usepackage[T1]{fontenc}
\usetikzlibrary{shapes.geometric, arrows.meta, calc, automata, positioning, fit, quotes}

\tikzstyle{startstop} = [rectangle, rounded corners, minimum width=3cm, minimum height=1cm, text centered, draw=black, fill=white]
\tikzstyle{io} = [trapezium, trapezium left angle=70, trapezium right angle=110, minimum width=3cm, minimum height=1cm, text centered, draw=black, fill=white]
\tikzstyle{process} = [rectangle, minimum width=3cm, minimum height=1cm, text centered, text width=3cm, draw=black, fill=white]
\tikzstyle{decision} = [diamond, minimum width=3cm, minimum height=1cm, text centered, draw=black, fill=white]
\tikzstyle{arrow} = [thick,->,>=stealth]
\tikzstyle{line} = [thick]

\begin{document}

\begin{tikzpicture}[node distance=2cm, auto]

\node (start) [startstop] {Proceso maestro Endurance};
\node (dec1) [decision, below of=start, aspect=2] {Lago de bufer > 1};
\node (pro1) [process, below of=dec1] {\shortstack{Extraer 2 objetos\\del buffer (a, b)}};
\node (pro2) [process, below of=pro1] {Requerir llanta};
\node (pro3) [process, below of=pro2] {\shortstack{Llamar a \\enllantado(a, b)}};
\node (pro4) [process, below of=pro3] {\shortstack{Llamar a \\enllantado(a, b)}};
\node (pro5) [process, below of=pro4] {Requerir maquina endurance};
\node (pro6) [process, below of=pro5] {Llamar a montado(a)};
\node (pro7) [process, below of=pro6] {Llamar a montado(b)};
\node (pro8) [process, below of=pro7] {\shortstack{Llamar a \\acondicionado(a, b)}};
\node (pro9) [process, below of=pro8] {\shortstack{Llamar a \\medir presion(a, b)}};
\node (pro10) [process, below of=pro9] {\shortstack{Llamar a \\Ensayo endurance\\(a, b)}};
\node (pro11) [process, below of=pro10] {\shortstack{Liberar \\Llanta y maquina}};

\draw [arrow] (dec1) -- (pro1);
\draw [arrow] (pro1) -- (pro2);
\draw [arrow] (pro2) -- (pro3);
\draw [arrow] (pro3) -- (pro4);
\draw [arrow] (pro4) -- (pro5);
\draw [arrow] (pro5) -- (pro6);
\draw [arrow] (pro6) -- (pro7);
\draw [arrow] (pro7) -- (pro8);
\draw [arrow] (pro8) -- (pro9);
\draw [arrow] (pro9) -- (pro10);
\draw [arrow] (pro10) -- (pro11);


\end{tikzpicture}


\end{document}
