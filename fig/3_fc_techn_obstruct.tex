\documentclass{standalone}
\usepackage[utf8]{inputenc}
\usepackage[spanish]{babel}
\usepackage{tikz}
\usetikzlibrary{shapes.geometric, arrows.meta, calc, automata, positioning, fit, quotes}

\tikzstyle{startstop} = [rectangle, rounded corners, minimum width=3cm, minimum height=1cm, text centered, draw=black, fill=white]
\tikzstyle{io} = [trapezium, trapezium left angle=70, trapezium right angle=110, minimum width=3cm, minimum height=1cm, text centered, draw=black, fill=white]
\tikzstyle{process} = [rectangle, minimum width=3cm, minimum height=1cm, text centered, text width=3cm, draw=black, fill=white]
\tikzstyle{decision} = [diamond, minimum width=3cm, minimum height=1cm, text centered, draw=black, fill=white]
\tikzstyle{arrow} = [thick,->,>=stealth]
\tikzstyle{line} = [thick]

\begin{document}

\begin{tikzpicture}[node distance=2cm, auto]

\node (start) [startstop] {Duración del turno = X};
\node (dec) [decision, below of=start] {Es viernes?};
\node (pro1) [process, below of=dec, yshift=-0.5cm] {Trascurrir X horas};
\node (pro2) [process, below of=pro1] {Obstruir técnico 24-X horas};
\node (pro3) [process, right of=dec, xshift=2cm] {Obstruir técnico 48 horas};

\coordinate [left of=pro2, xshift=-1cm] (ur1);

\draw [arrow] (dec) -- node [near start] {No} (pro1);
\draw [arrow] (dec) -- node [near start] {Si} (pro3);
\draw [arrow] (pro3) |- (pro2);
\draw [arrow] (pro1) -- (pro2);
\draw [line] (pro2) -- (ur1);
\draw [arrow] (ur1) |- (dec);


\end{tikzpicture}


\end{document}
