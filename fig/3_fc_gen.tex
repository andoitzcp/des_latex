\documentclass{standalone}
\usepackage[utf8]{inputenc}
\usepackage[spanish]{babel}
\usepackage{tikz}
\usetikzlibrary{shapes.geometric, arrows.meta, calc, automata, positioning, fit, quotes}

\tikzstyle{startstop} = [rectangle, rounded corners, minimum width=3cm, minimum height=1cm, text centered, draw=black, fill=white]
\tikzstyle{io} = [trapezium, trapezium left angle=70, trapezium right angle=110, minimum width=3cm, minimum height=1cm, text centered, draw=black, fill=white]
\tikzstyle{process} = [rectangle, minimum width=3cm, minimum height=1cm, text centered, text width=3cm, draw=black, fill=white]
\tikzstyle{decision} = [diamond, minimum width=3cm, minimum height=1cm, text centered, draw=black, fill=white]
\tikzstyle{arrow} = [thick,->,>=stealth]
\tikzstyle{line} = [thick]

\begin{document}

\begin{tikzpicture}[node distance=2cm, auto]

	\node (start) [startstop] {\shortstack{Atributos \\ \\ ID=i, RUN=r, TEST=t}};
\node (pro1) [process, below of=start] {i += 1};
\node (pro2) [process, below of=pro1] {\shortstack{Crear \\ objeto(i, r, t)}};
\node (pro3) [process, below of=pro2] {\shortstack{Añadir \\ objeto a buffer}};
\node (pro4) [process, below of=pro3] {\shortstack{Llamar a \\ proceso maestro}};
\node (pro5) [process, below of=pro4] {Transcurrir tiempo entre llegadas};

\coordinate [left of=pro5, xshift=-1cm] (ur1);

\draw [arrow] (pro1) -- (pro2);
\draw [arrow] (pro2) -- (pro3);
\draw [arrow] (pro3) -- (pro4);
\draw [arrow] (pro4) -- (pro5);
\draw [arrow] (pro5) -- (ur1);
\draw [arrow] (ur1) |- (pro1);


\end{tikzpicture}


\end{document}
