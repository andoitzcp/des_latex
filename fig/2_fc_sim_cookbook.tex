\documentclass{standalone}
\usepackage[utf8]{inputenc}
\usepackage[spanish]{babel}
\usepackage{tikz}
\usetikzlibrary{shapes.geometric, arrows.meta, calc, automata, positioning, fit, quotes}

\tikzstyle{startstop} = [rectangle, rounded corners, minimum width=3cm, minimum height=1cm, text centered, draw=black, fill=white]
\tikzstyle{io} = [trapezium, trapezium left angle=70, trapezium right angle=110, minimum width=3cm, minimum height=1cm, text centered, draw=black, fill=white]
\tikzstyle{process} = [rectangle, minimum width=3cm, minimum height=1cm, text centered, text width=3cm, draw=black, fill=white]
\tikzstyle{decision} = [diamond, minimum width=3cm, minimum height=1cm, text centered, draw=black, fill=white]
\tikzstyle{arrow} = [thick,->,>=stealth]
\tikzstyle{line} = [thick]

\begin{document}

\begin{tikzpicture}[node distance=2cm, auto]

\node (start) [startstop] {Formulación del problema};
\node (pro1) [process, below of=start] {Establecimiento de objetivos y plan general};
\node (pro2a) [process, below of=pro1, xshift=-3cm] {Conceptualización del modelo};
\node (pro2b) [process, below of=pro1, xshift=3cm] {Recopilación de datos};
\node (pro3) [process, below of=pro2a, xshift=3cm] {Traducción del modelo};
\node (dec1) [decision, below of=pro3,aspect=2, yshift=0cm] {¿Verificado?};
\node (dec2) [decision, below of=dec1,aspect=2, yshift=0cm] {¿Validado?};
\node (pro4) [process, below of=dec2, yshift=-0.25] {Diseño experimental};
\node (pro5) [process, below of=pro4] {Ejecuciones de producción y análisis};
\node (dec3) [decision, below of=pro5, aspect=3, yshift=0cm] {¿Más ejecuciones?};
\node (pro6) [process, below of=dec3, yshift=-0.2cm] {Documentación y comunicación};
\node (stop) [startstop, below of=pro6, yshift=0.2cm] {Implementación};

\coordinate [below of=pro1, yshift=1cm] (ur1);
\coordinate [left of=pro2a] (ur2a);
\coordinate [right of=pro2b] (ur2b);
\coordinate [below of=pro2a, xshift=3cm, yshift=1cm] (ur3);
\coordinate [left of=pro3] (ur4);
\coordinate [left of=pro4, xshift=-1cm] (ur5);
\coordinate [right of=pro5, xshift=1cm] (ur6);

\draw [arrow] (start) -- (pro1);
\draw [arrow] (pro1) -- (ur1);
\draw [arrow] (ur1) -| (pro2a);
\draw [arrow] (ur1) -| (pro2b);
\draw [arrow] (pro2a) |- (ur3);
\draw [arrow] (pro2b) |- (ur3);
\draw [arrow] (ur3) -- (pro3);
\draw [arrow] (pro3) -- (dec1);
\draw [line] (dec1) -| node [near start] {No} (ur4);
\draw [arrow] (ur4) -- (pro3);
\draw [line] (dec2) -| node [very near start, below] {No} (ur2a);
\draw [arrow] (ur2a) -- (pro2a);
\draw [line] (dec2) -| node [very near start, below] {No} (ur2b);
\draw [arrow] (ur2b) -- (pro2b);
\draw [arrow] (dec1) -- node [near start] {Si} (dec2);
\draw [arrow] (dec2) -- node [near start] {Si} (pro4);
\draw [arrow] (pro4) -- (pro5);
\draw [arrow] (pro5) -- (dec3);
\draw [line] (dec3) -| node [very near start, below] {Si} (ur5);
\draw [arrow] (ur5) -- (pro4);
\draw [line] (dec3) -|node [very near start, below] {Si} (ur6);
\draw [arrow] (ur6) -- (pro5);
\draw [arrow] (dec3) -- node [near start] {No} (pro6);
\draw [arrow] (pro6) -- (stop);

\end{tikzpicture}


\end{document}
