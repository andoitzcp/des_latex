\documentclass{standalone}
\usepackage[utf8]{inputenc}
\usepackage[spanish]{babel}
\usepackage{tikz}
\usetikzlibrary{shapes.geometric, arrows.meta, calc, automata, positioning, fit, quotes}

\tikzstyle{startstop} = [rectangle, rounded corners, minimum width=3cm, minimum height=1cm, text centered, draw=black, fill=white]
\tikzstyle{io} = [trapezium, trapezium left angle=70, trapezium right angle=110, minimum width=3cm, minimum height=1cm, text centered, draw=black, fill=white]
\tikzstyle{process} = [rectangle, minimum width=3cm, minimum height=1cm, text centered, text width=3cm, draw=black, fill=white]
\tikzstyle{decision} = [diamond, minimum width=3cm, minimum height=1cm, text centered, draw=black, fill=white]
\tikzstyle{arrow} = [thick,->,>=stealth]
\tikzstyle{line} = [thick]

\begin{document}

\begin{tikzpicture}[node distance=4cm, auto]

\node (st1) [startstop] {Determinar objetivos};
\node (st2) [process, below of=st1, yshift=2cm] {Identificar y priorizar preguntas clave};
\node (st3) [process, right of=st2] {Definir outputs necesarios};
\node (st4) [process, right of=st3] {Definir alcance del modelo y detalle};
\node (st5) [startstop, above of=st4, yshift=-2cm] {Especificar inputs};

\draw [arrow] (st1) -- (st2);
\draw [arrow] (st2) -- (st3);
\draw [arrow] (st3) -- (st4);
\draw [arrow] (st4) -- (st5);



\end{tikzpicture}


\end{document}
