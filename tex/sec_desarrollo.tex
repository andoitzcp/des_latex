% **************************************************************************** %
%                                                                              %
%                                                         :::      ::::::::    %
%    sec_desarrollo.tex                                 :+:      :+:    :+:    %
%                                                     +:+ +:+         +:+      %
%    By: acampo-p <acampo-p@student.42urduliz.com>  +#+  +:+       +#+         %
%                                                 +#+#+#+#+#+   +#+            %
%    Created: 2022/12/07 17:42:28 by acampo-p          #+#    #+#              %
%    Updated: 2023/02/13 04:11:05 by acampo-p         ###   ########.fr        %
%                                                                              %
% **************************************************************************** %

\section{DESARROLLO}

\subsection{FORMULACIÓN DEL PROBLEMA, \newline ESTABLECIMIENTO DE OBJETIVOS Y PLAN \newline GENERAL}

Esta etapa corresponde a la sección~\ref{sec:intro} de introducción.

\subsection{CONCEPTUALIZACIÓN DEL MODELO}

Como se menciona en la sección~\ref{sec:modelconcept}, a la hora de diseñar el modelo
es conveniente comenzar desde el planteamiento de los objetivos,
e ir adquiriendo detalle hasta llegar a definir los \textit{inputs}.

Una vez han quedado los objetivos descritos,
se ha procedido ha determinar las cuestiones clave para dar con una solución.
La siguiente lista de preguntas ha sido ha sido formulada.

\begin{itemize}
	\item ¿Que sucedería si el tiempo disponible para el montaje de las maquinas fuese mayor? 
	\item ¿Cual sería el efecto de que dos técnicos trabajasen simultáneamente en los ensayos?
	\item ¿Como afectaría la instalación de maquinas Endurance
		adicionales a la cantidad de ensayos realizados?
\end{itemize}

Para responder a las preguntas planteadas,
se ha programado la simulación de tal manera que devuelva una tabla
conteniendo los tiempos de inicio y fin de cada proceso.
A través de análisis de los datos indexados,
se han podido obtener estadísticas como:
numero de ensayos totales realizados por tipo de test,
o el tiempo total de maquina en funcionamiento
por cada ejecución de la simulación.

El modelo a reproducir para obtener los datos descritos	en el anterior párrafo,
ha sido deliberadamente diseñado de manera simple.
Esto ha facilitado la obtención de datos representativos
y su posterior análisis.
Se ha tomado parte de el diagrama de la Figura~\ref{fig:2_fc_lep_diagram},
Excluyendo ciertos procesos irrelevantes
para las preguntas previamente formuladas.
Estos diagramas se encuentran en el Anexo~\ref{apnd3}.

Los \textit{inputs} necesarios para el funcionamiento del modelo
se han recopilado en las Tablas~\ref{tab:3_tbl_sim_det},
\ref{tab:3_tbl_rsrc},
\ref{tab:3_tbl_indoor_det}
y \ref{tab:3_tbl_rr_det}.

\begin{table}[H]
	\centering
	\caption{Características generales de la simulación.}
	\documentclass[varwidth=\maxdimen]{standalone}
\usepackage[utf8]{inputenc}
\usepackage[spanish]{babel}
\usepackage{booktabs}

\begin{document}

\begin{tabular}{ l c }
	\toprule
	Propiedad	& Valor  \\
	\midrule
	Duración		& 365 días \\
	Jornada laboral	& 8 h \\
	Turno(s)		& 1 \\
	Ejecuciones		& 100 unds. \\
	\bottomrule
\end{tabular}

\end{document}

	\label{tab:3_tbl_sim_det}
\end{table}

\begin{table}[H]
	\centering
	\caption{Recursos representados en el modelo.}
	\documentclass[varwidth=\maxdimen]{standalone}
\usepackage[utf8]{inputenc}
\usepackage[spanish]{babel}
\usepackage{booktabs}

\begin{document}

\begin{tabular}{ l c }
	\toprule
	Recurso & Cantidad (unds.) \\
	\midrule
	Máquina \textit{Endurance}			& 4 \\
	Máquina \textit{Rolling	Resistance}	& 1 \\
	Máquina enllantado					& 1 \\
	Personal técnico					& 1 \\
	Llantas \textit{Endurance}			& 14 \\
	Llantas \textit{Rolling	Resistance}	& 20 \\
	\bottomrule
\end{tabular}

\end{document}

	\label{tab:3_tbl_rsrc}
\end{table}

\begin{table}
	\centering
	\caption{Características del proceso de ensayos endurance.}
	\documentclass[varwidth=\maxdimen]{standalone}
\usepackage[utf8]{inputenc}
\usepackage[spanish]{babel}
\usepackage{booktabs}

\begin{document}

\begin{tabular}{ l c c }
	\toprule
	Proceso	& Prioridad	& Distribución (min)\\
	\midrule
	Tiempo entre llegadas	& -	& Exponencial \\
							&	& $\lambda=120$ \\
										 \\
	Enllantado			& 0		& Triangular \\
						&		& $a=12$, $m=15$, $b=20$ \\
										 \\
	Montaje				& -1	& Normal \\
						&		& $\mu=45$, $\sigma=10$ \\
										 \\
	Acondicionamiento	& -3	& Constante \\
						&		& $c=180$ \\
										 \\
	Ajuste de presión	& -4	& Constante  \\
						&		& $c=5$ \\
										 \\
	Ensayo				& -		& Normal \\
						&		& $\mu=4320$, $\sigma=360$ \\
	\bottomrule
\end{tabular}

\end{document}

	\label{tab:3_tbl_indoor_det}
\end{table}

\begin{table}
	\centering
	\caption{Características del proceso de ensayos rolling.}
	\documentclass[varwidth=\maxdimen]{standalone}
\usepackage[utf8]{inputenc}
\usepackage[spanish]{babel}
\usepackage{booktabs}

\begin{document}

\begin{tabular}{ l c c }
	\toprule
	Proceso	& Prioridad	& Distribución (min) \\
	\midrule
	Tiempo entre llegadas	& -		& Exponencial (min) \\
							&		& $\lambda=48$ \\
							\\
	Enllantado				& 0		& Triangular \\
							&		& $a=12$, $m=15$, $b=20$ \\
							\\
	Acondicionamiento		& -2	& Constante \\
							&		& $c=360$ \\
							\\
	Montaje					& -2	& Normal \\
							&		& $\mu=45$, $\sigma=10$ \\
							\\
	Ensayo					& -		& Normal \\
							&		& $\mu=180$, $\sigma=20$ \\
	\bottomrule
\end{tabular}

\end{document}

	\label{tab:3_tbl_rr_det}
\end{table}


\subsection{RECOPILACIÓN DE DATOS}

En un proyecto de un laboratorio real,
primeramente se habría buscado información en las bases de datos y en los registros históricos disponibles.
En caso de no disponer de estos,
se habrían medido la duración de los procesos en el acto.
Una de las propuestas es cronometrar cada proceso entre 20 y 30 veces
para obtener un rango de tiempo estimado.
A este rango, se le aplicaría un
ajuste de distribución de probabilidad que se usaría en la DES.
Al trabajar con un laboratorio ficticio,
no es posible la recopilación de datos dentro de un laboratorio.
No obstante, el proyecto se ha planteado en base a estimaciones fundamentadas.
Se ha tratado de representar cada proceso mediante
distribuciones acordes a sus propiedades,
descritas en la sección~\ref{sec_prob_dist}.

Para los tiempos entre llegadas de objetos en la simulación,
se ha elegido la distribución exponencial debido a
su asociación a tiempos de espera entre eventos.

Para los procesos de los que menos información se poseía,
como puede ser los procesos de montaje y enllantado,
se ha optado por representarlos mediante una distribución triangular.
Como se indica en la sección~\ref{sec_prob_dist},
las distribuciones triangulares son adecuadas para aquellos procesos en los que no se disponga de mucha información.

Finalmente en el caso de los ensayos \textit{Endurance},
se ha estimado su duración media en base a los estándares de calidad
especificados en la norma UNECE-54.
En el caso de los ensayos \textit{Rolling Resistance} se ha estimado en base al estándar SAE J1269.
La duración de estos ensayos,
se ha tratado de representar mediante distribuciones normales,
debido a la sencillez y versatilidad de las mismas.

\subsection{TRADUCCIÓN DEL MODELO}

El modelo traducido, que se encuentra en el Anexo~\ref{apnd3},
puede dividirse en 3 procesos generales:

\paragraph{La generación de entidades}
se encarga de crear instancias de las cubiertas que son alimentadas
a los procesos maestros de cada ensayo.
Son 2 los generadores que posee la simulación,
uno para los ensayos Endurance, y otro para los ensayos Rolling.

\paragraph{Los procesos maestros}
son los responsables de ejecutar los subprocesos
en el orden correspondiente mientras demandan
los recursos necesarios durante su transcurso.
Al igual que los generadores, hay 2 procesos maestros,
uno para cada tipo de ensayo.

\paragraph{Los subprocesos}
son las subrutinas llamadas por los procesos maestros.
Crear módulos para cada tarea mejora la legibilidad del código.

\paragraph{La obstrucción del técnico}
es el proceso que simula el inicio y el fin de una jornada laboral,
y hace transcurrir el tiempo, incluyendo los fines de semana.

\subsection{VERIFICACIÓN Y VALIDACIÓN}

El proceso de verificación se ha realizado a lo largo de la simulación.
Se ha comprobado que los pasos se ejecutan en el orden esperado,
y se ha revisado la lógica de los algoritmos propuestos.

Respecto al proceso de validación,
se ha tomado como referencia el recuento ensayos realizados durante un año,
y el tiempo de funcionamiento de maquina, y se ha comprobado que fueran similares. 
