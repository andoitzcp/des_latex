% **************************************************************************** %
%                                                                              %
%                                                         :::      ::::::::    %
%    sec_conclusiones.tex                               :+:      :+:    :+:    %
%                                                     +:+ +:+         +:+      %
%    By: acampo-p <acampo-p@student.42urduliz.com>  +#+  +:+       +#+         %
%                                                 +#+#+#+#+#+   +#+            %
%    Created: 2022/12/07 17:42:28 by acampo-p          #+#    #+#              %
%    Updated: 2023/02/08 18:50:14 by acampo-p         ###   ########.fr        %
%                                                                              %
% **************************************************************************** %

\section{CONCLUSIONES}

A modo de cierre de este trabajo,
se puede señalar que se llegó a lograr el objetivo planteado
al comienzo del proyecto.
Dentro de las posibilidades tomadas en cuenta,
el escenario mas óptimo ha sido hallado.
Mantener un enfoque hacia los hitos marcados al principio,
ha ayudado en el extenso proceso de la elaboración de la simulación.

Entre los resultados más relevantes obtenidos,
se encuentran la gran capacidad de mejora que posee el laboratorio
en su estado actual.
El descubrimiento de una tendencia proporcional
entre la cantidad de horas trabajadas
y el numero de ensayos realizados, ha sido particularmente útil.
Mediante los resultados de la regresión obtenida,
se podrá facilitar la futura organización de los turnos de trabajo,
y anticipar las desviaciones respecto a los objetivos de manera más precisa.
El descarte de la ampliación de la maquinaria del laboratorio,
supondrá un ahorro tanto en espacio como en inversión de la fabrica.

Respecto al proceso, se han encontrado numerosas dificultades.
Primeramente, la familiarización con las DES,
ha requerido un extenso trabajo de investigación
acerca de sus métodos e implementaciones.
Se ha requerido ajustar este método de simulación a los objetivos del trabajo,
lo que ha supuesto un periodo previo al desarrollo,
en el que se han valorado múltiples enfoques.
La tarea más complicada, ha resultado ser
el proceso de traducción del modelo desde los diagramas, a el código fuente.
Este periodo, ha durado semanas,
en las cuales se ha aprendido cero a usar la librería Simpy.
El proyecto, al involucrar tantas lineas de código,
ha supuesto un proceso complicado de
detección y corrección de errores de funcionamiento,
que ha limitado lo que podría haber sido el análisis final.
Aún así, el análisis de los resultados ha resultado satisfactorio,
para un Trabajo de Fin de Grado.

Este trabajo ha causado, de manera transversal,
un entendimiento aún mas detallado del proceso de lo que se tenía en un inicio.
Además, los conocimientos obtenidos en este TFG,
podrán ser útiles en futuros procesos de optimización en producción.
